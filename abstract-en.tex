\begin{mdframed}[linecolor=Prune,linewidth=1]

\textbf{Title:} Geographic and socio-demographic disparities in oncology care pathways.

\noindent \textbf{Keywords:} 3 à 6 mots clefs (version en anglais)

\vspace{-.5cm}
\begin{multicols}{2}
\noindent \textbf{Abstract:} In France, during year 2018, there was 382,000 new cancer cases and 157,400 deaths. The 2014-2019 Cancer Plan sets new objectives for cancer care in France. In particular, objectives 2 and 7 emphasize the quality of the care pathways: they aim respectively to guarantee the quality and safety of care and ensure comprehensive and personalized care. In order to standardize the care pathways while personalizing care, care trajectories have been introduced. The definition of these optimal care trajectories is based on national and international good practice recommendations. We propose to study in detail the geographical and socio-demographic disparities in the care pathways of cancer patients in France. First, we will try to characterize the health care institutions in France based on their oncology activity. Then, we will study the distribution of these centers on the territory, in order to highlight possible disparities in access to them. Finally, we will try to propose a care center recommendation algorithm, using the French social security database (SNDS). This algorithm will aim at guiding patients towards the optimal center, in order to maximize the quality of the care pathways.

\end{multicols}
\end{mdframed}