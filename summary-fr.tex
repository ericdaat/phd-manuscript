Le cancer est l'une des principales causes de mortalité dans le monde,
représentant près de 10 millions de décès en 2020. Selon l'Organisation mondiale
de la santé, une personne sur cinq dans le monde développe un cancer au cours de
sa vie. Les développements importants des traitements oncologiques observés ces
dernières années ont amélioré les résultats pour les patients atteints de
cancer. Bien que ces progrès aient un impact positif, ils ont augmenté la
complexité de la prestation des soins. Pour faire face aux défis posés par cette
complexité, des parcours de soins ont été introduits. Dans la littérature, les
parcours de soins ont été définis comme « une intervention complexe pour la
prise de décision mutuelle et l'organisation des processus de soins pour un
groupe bien défini de patients pendant une période bien définie ». Un parcours
de soins vise à renforcer la qualité des soins en améliorant les résultats des
patients, en augmentant leur satisfaction et en optimisant l'utilisation des
ressources. La littérature fait état de multiples preuves de disparités dans les
parcours de santé et de soins, dont certaines sont dues à des facteurs externes
tels que le statut socio-économique ou le lieu de résidence. Par exemple, le
statut socioéconomique, reflété par le revenu, l'éducation ou la profession,
exacerbe les problèmes de santé, y compris le cancer. En France, l'Institut
national du cancer (INCA) est l'agence d'État pour l'expertise sanitaire et
scientifique en cancérologie, chargée de coordonner les actions de lutte contre
le cancer. Depuis 2003, l'INCA produit des rapports contenant des
recommandations nationales et des mesures visant à mobiliser les acteurs de
santé publique autour de la prévention, du dépistage, de l'organisation des
soins, de la recherche, du soutien aux patients et à leurs familles, et de
l'après-cancer. À ce jour, trois plans cancer ont été publiés et le dernier
couvrait la période 2014-2019. Ce plan est largement focalisé sur les inégalités
de prise en charge en oncologie, avec pour objectifs d'accroître les
connaissances sur cette question et de lutter contre ce problème par des
interventions concrètes. Les données de vie réelle des patients représentent un
volume d'informations sans précédent, actuellement sous-exploité. En
particulier, en France, la sécurité sociale génère une grande base de données
structurée à des fins administratives : le Système National des Données de Santé
(SNDS). Le SNDS rassemble des données administratives complètes et actualisées
sur 98\% de la population française. L'exploitation du SNDS, à des fins de
recherche, est une opportunité exceptionnelle d'élargir le champ de la recherche
à l'amélioration des parcours de soins. L'objectif de ce travail est d'exploiter
les données de vie réelles des patients pour fournir des mesures et des outils
permettant de lutter contre les disparités dans les parcours de soins en
oncologie, en France. Nous avons choisi d'aborder en les disparités
géographiques et socio-démographiques, et nous ne nous sommes pas concentrés sur
un site de cancer spécifique. La principale source de données utilisée a été la
base de données du PMSI, pour accéder aux données des hôpitaux et étudier les
parcours de soins des patients. Nous avons limité l'analyse à l'année 2018, et
n'avons pas étudié l'impact de la pandémie de COVID dans les parcours de soins.
Chaque métrique et outil que nous avons développé au cours de cette thèse pourra
être réutilisé dans d'autres travaux de recherche. Nous avons tout d’abord
proposé une caractérisation de chaque centre de soins en France en termes de
spécialisation oncologique. Ce label oncologique aidera les médecins, les
patients, les chercheurs ou les professionnels de la santé publique à mieux
évaluer les hôpitaux et leur répartition spatiale dans le pays. Deuxièmement,
nous avons calculé un score d'accessibilité à l'oncologie, pour identifier les
zones où les hôpitaux spécialisés en oncologie sont rares. Troisièmement, nous
avons proposé un algorithme d'optimisation pour cibler les hôpitaux qui
devraient être développés en priorité pour améliorer cette accessibilité.
Quatrièmement, nous avons étudié les déplacements des patients entre leur
commune de résidence et les hôpitaux qu'ils visitent. Nous avons développé un
indice de la charge de déplacement pour mesurer non seulement le déplacement en
tant que distance, mais aussi en tant que combinaison de la distance, de la
durée et de la sinuosité de la route. Nous avons également estimé l'empreinte
carbone des déplacements de ces patients et simulé un scénario dans lequel
chaque patient se rendrait au centre spécialisé le plus proche. Nous pensons
qu'une plus grande transparence dans les soins oncologiques pourrait bénéficier
aux patients et les aider, ainsi que leur médecin, à trouver l'hôpital le plus
adapté situé à une distance raisonnable. Ainsi, nous avons construit une
application web qui répertorie toutes les caractéristiques des hôpitaux, à la
fois à destination des patients et des médecins. Enfin, nous avons développé un
algorithme d'allocation basé sur le transport optimal pour diriger les patients
vers un hôpital proche et adapté. Cependant, nous n'avons testé ce modèle que
sur des données synthétiques, des recherches supplémentaires sont nécessaires
pour l'appliquer à des données réelles de parcours de soins.
