\chapter{Introduction}

There will be an estimated 382,000 new cases of incident cancer and 157,400 deaths in 2018 in France. The French Cancer Plan \cite{buzyn_plan_2014} 2014-2019 announces the objectives to be implemented in the fight against cancer in France. In particular, objectives 2 and 7 insist on the quality of the care pathway: they aim respectively to ``guarantee the quality and safety of care'' and ``ensure comprehensive and personalized care''.

\section{Care pathways}

In order to standardize the care pathway while personalizing management, care trajectories have been established. The definition of these optimal care trajectories is based on national and international good practice recommendations.

\section{Disparities in care pathways}

\ac{inca} has published several studies comparing care pathways in France with national and international recommendations. One of them concerns the time required for the management of breast and lung cancer. This study found differences according to the status of the institution of first therapeutic management or the region \cite{bernard_ledesert_etude_2012}. However, the study does not explain these differences, in particular because of the lack of availability of socio-demographic indicators. Today, there are several public data sources that provide access to these indicators at the municipality level. Incorporating additional data sources could lead to better understanding of where these disparities in management come from according to the care facilities. Indeed, the multiplicity of care centers, their types, the distance of the care centers from the patients' homes and the location of treatments are all factors that can degrade the care pathway and impact the prognosis of cancer patients.

\subsection*{Geographical disparities}

A study analyzed the care pathways in Tanzania for patients with tuberculosis \cite{mhalu_pathways_2019}. The study highlights the complexity of the pathways from the first symptoms to diagnosis and the high cost of accessing health care facilities.

Several studies have investigated the optimal distribution of care centers in different countries, as well as their accessibility by road or public transport. A study of health care facilities for the city of Shenzhen in China showed differences in access to health care depending on the mode of transport used. It appears that public transport users are at a disadvantage compared to patients with a car \cite{tao_spatial_2018}. Mandel et al \cite{mandel_optimizing_2018} showed that an application similar to Google Maps for guiding patients to different care centers in a multi-site hospital reduces patient travel time. In particular, the application uses real-time traffic data for referral. Jia et al \cite{jia_selecting_2014} proposed a method to select the optimal care center using several criteria such as geographic accessibility and service quality. In particular, transportation networks such as high-speed lines and highways are taken into account in the center selection.

\subsection*{Socio-demographic disparities}

Various studies have investigated the impact of socio-demographic factors on the course of care and survival prognosis of patients. An American study on lung cancer patients showed that good physical and intellectual conditions were linked to better survival from the disease \cite{pierzynski_socio-demographic_2018}.

A second study shows differences in access to care related to ethnicity for patients with psychosis \cite{anderson_meta-analysis_2014}.

Even though the burden of cancer is easing in the United States, the decline is unequal among different racial, ethnic and socio-economic groups \cite{viswanath_science_2005}.

The aim of this study was to examine the impact of patient demographics, tumor characteristics, and treatment type on time to treatment (TTT) in patients with breast cancer treated at a safety net medical center with a diverse patient population. Longer median TTT was noted for Black and single patients \cite{khanna_impact_2017}.

\subsection*{Gender-related disparities}

Gender appears to have an impact on care pathways. For example, men may have difficulty talking about their symptoms, fearing that it will be perceived as a sign of weakness; whereas women who require care are more likely to be neglected \cite{ferrari_gender_2018}. Indeed, women with myocardial infarction have a higher mortality rate than men, and this discrepancy appears to be partially due to delayed diagnosis and access to appropriate care \cite{bugiardini_delayed_2017}. Similarly, a pediatric study of kidney transplantation showed that young girls had less rapid access to transplantation than young boys. This is partly due to non-medical reasons such as parental and practitioner behavior regarding organ donation \cite{hogan_j_gender_2016}.
More specifically, the gender of the patient could have an impact on the oncology care pathway. Indeed, several studies show that women's treatment for several types of cancers is suboptimal. This would at least partially explain why their chances of survival from these diseases are lower than those of men \cite{park_a_undertreatment_2019,carter_paulson_e_gender_2009,rose_sex_2016}.
The above examples suggest that patient survival could be improved by taking gender into consideration in the care pathway. However, at present, gender differences in the oncology care pathway are barely explored.
