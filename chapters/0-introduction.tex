\chapter{Introduction}

\section{Cancer epidemiology}

Cancer is a leading cause of death worldwide.

\section{Evidences of disparities in cancer care}

\subsection{Geographical disparities}

A study analyzed the care pathways in Tanzania for patients with tuberculosis
\cite{mhalu_pathways_2019}. The study highlights the complexity of the pathways
from the first symptoms to diagnosis and the high cost of accessing health care
facilities.

Several studies have investigated the optimal distribution of care centers in
different countries, as well as their accessibility by road or public transport.
A study of health care facilities for the city of Shenzhen in China showed
differences in access to health care depending on the mode of transport used. It
appears that public transport users are at a disadvantage compared to patients
with a car \cite{tao_spatial_2018}. Mandel et al \cite{mandel_optimizing_2018}
showed that an application similar to Google Maps for guiding patients to
different care centers in a multi-site hospital reduces patient travel time. In
particular, the application uses real-time traffic data for referral. Jia et al
\cite{jia_selecting_2014} proposed a method to select the optimal care center
using several criteria such as geographic accessibility and service quality. In
particular, transportation networks such as high-speed lines and highways are
taken into account in the center selection.

\subsection{Socio-demographic disparities}

Various studies have investigated the impact of socio-demographic factors on the
course of care and survival prognosis of patients. An American study on lung
cancer patients showed that good physical and intellectual conditions were
linked to better survival from the disease
\cite{pierzynski_socio-demographic_2018}.

A second study shows differences in access to care related to ethnicity for
patients with psychosis \cite{anderson_meta-analysis_2014}.

Even though the burden of cancer is easing in the United States, the decline is
unequal among different racial, ethnic and socio-economic groups
\cite{viswanath_science_2005}.

The aim of this study was to examine the impact of patient demographics, tumor
characteristics, and treatment type on time to treatment (TTT) in patients with
breast cancer treated at a safety net medical center with a diverse patient
population. Longer median TTT was noted for Black and single patients
\cite{khanna_impact_2017}.

\subsection{Gender-related disparities}

Gender appears to have an impact on care pathways. For example, men may have
difficulty talking about their symptoms, fearing that it will be perceived as a
sign of weakness; whereas women who require care are more likely to be neglected
\cite{ferrari_gender_2018}. Indeed, women with myocardial infarction have a
higher mortality rate than men, and this discrepancy appears to be partially due
to delayed diagnosis and access to appropriate care
\cite{bugiardini_delayed_2017}. Similarly, a pediatric study of kidney
transplantation showed that young girls had less rapid access to transplantation
than young boys. This is partly due to non-medical reasons such as parental and
practitioner behavior regarding organ donation \cite{hogan_j_gender_2016}. More
specifically, the gender of the patient could have an impact on the oncology
care pathway. Indeed, several studies show that women's treatment for several
types of cancers is suboptimal. This would at least partially explain why their
chances of survival from these diseases are lower than those of men
\cite{park_a_undertreatment_2019,carter_paulson_e_gender_2009,rose_sex_2016}.
The above examples suggest that patient survival could be improved by taking
gender into consideration in the care pathway. However, at present, gender
differences in the oncology care pathway are barely explored.

\section{Disparities in cancer care in France}

The French Cancer Plan \cite{buzyn_plan_2014} 2014-2019 announces the objectives
to be implemented in the fight against cancer in France. In particular,
objectives 2 and 7 insist on the quality of the care pathway: they aim
respectively to ``guarantee the quality and safety of care'' and ``ensure
comprehensive and personalized care''. In order to standardize the care pathway
while personalizing management, care trajectories have been established. The
definition of these optimal care trajectories is based on national and
international good practice recommendations.

\ac{inca} has published several studies comparing care pathways in France with
national and international recommendations. One of them concerns the time
required for the management of breast and lung cancer. This study found
differences according to the status of the institution of first therapeutic
management or the region \cite{bernard_ledesert_etude_2012}. However, the study
does not explain these differences, in particular because of the lack of
availability of socio-demographic indicators. Today, there are several public
data sources that provide access to these indicators at the municipality level.
Incorporating additional data sources could lead to better understanding of
where these disparities in management come from according to the care
facilities. Indeed, the multiplicity of care centers, their types, the distance
of the care centers from the patients' homes and the location of treatments are
all factors that can degrade the care pathway and impact the prognosis of cancer
patients.

\section{Context}

\subsection{Hospitals characterization}

Countries, such as the UK, USA and Canada, have been implementing a policy of
centralizing the care of patients for many specialized services
\cite{kelly_are_2016}. With such policy, patients are directed to a limited
number of hospitals with higher volumes and more specialized surgeons. There is
evidence that this process will have a positive impact on the health outcomes of
those patients treated in these specialized centres. For instance, centralized
care is beneficial for patients undergoing high-risk procedures, these surgeries
have lower mortality rates when performed by high-volume surgeons
\cite{pekala_centralization_2021,birkmeyer_surgeon_2003,finks_trends_2011,
    hollenbeck_getting_2007,goossens-laan_systematic_2011}. A centralized service
for ovarian cancer may lead to better survival outcomes; evidence from various
other sources suggests that this may also be more cost-effective
\cite{woo_centralisation_2012}. With the rural exodus, the sparsely populated
areas expanded, and several hospitals are serving relatively small populations.
As a result, surgeons operating in these facilities are managing fewer cases of
a given disease. For instance, in the South West of England, surgeons treating
epithelial ovarian cancer were managing fewer than ten cases of ovarian cancer
per year. There is a need to maintain a critical volume of work in order to
sustain surgical expertise \cite{olaitan_surgical_2001}. Through all these
evidences, it is clear that not all the hospitals are equal for cancer
treatment. In France, there are many hospitals that do not have the same degree
of oncology specialization. Hospitals are classified into different legal
categories like public hospitals or private structures, but there is no
indicator to assess the degree of oncology specialization and how large the
hospital is.

\subsection{Spatial accessibility}

% Accessibility
While a lot of the ongoing research is focusing on finding new cancer
treatments, accessibility to oncology care receives less attention. Yet, several
studies have showed that access to health services plays a key role in cancer
survival. For instance, geographic residency status and social environment seem
to explain treatment and prognosis disparities for patients with non-small cell
lung cancer \cite{johnson_treatment_2014}. In France, increases in travel times
to health services were associated with lower survival rates for patients with a
colorectal cancer \cite{dejardin_influence_2014}. In New Zealand, living in
deprived areas, far from a cancer center or from primary care was associated
with lower survival chances for patients with colorectal, lung and prostate
cancers \cite{haynes_cancer_2008}. Accessibility refers to the relative ease by
which services can be reached from a given location
\cite{wang_measurement_2012}. Accessibility can be defined by spatial factors,
determined by where you are; and non-spatial factors, determined by who you are
\cite{khan_integrated_1992}. In what follows, we restrict accessibility to
\acf{sa} and use both terms interchangeably. \ac{sa} methods assess the
availability of supply locations from demand locations, connected by a travel
impedance metric. Supply locations are characterized by their capacity or
quantity of available resource. Similarly, demand locations are characterized by
their population. Such methods have been successfully used to measure access to
healthcare, such as primary care \cite{guagliardo_spatial_2004} or oncology care
\cite{wang_measurement_2012,zahnd_spatial_2021,alahmadi_spatial_2013} in several
countries including France
\cite{launay_methodology_2019,gusmano_disparities_2014,gao_assessment_2016}.
When measuring accessibility for healthcare, the supply locations are often
physicians locations, whose capacity might be the number of physicians at that
location. Population locations represent where patients live. This could be the
precise address or a municipality. However, while accessibility to primary care
have been described in several studies, there is little work that focused on
oncology care specifically.

% Accessibility optimization
Uneven distributions of population and health-care providers lead to geographic
disparity in accessibility for patients \cite{wang_why_2020}, illustrated by our
previous results on accessibility. Several methods have been developed to
address these disparities. Location-allocation algorithms
\cite{church_location_1999} can optimize the distribution and supply of health
providers to reduce accessibility disparities. These algorithms seek the optimal
placement of facilities for a desirable objective under certain constraints
\cite{wang_measurement_2012}. For instance, an optimization algorithm  was
developed to improve the healthcare planning in rural China by finding the best
place and capacity for new health facilities \cite{luo_integrating_2014}. A
spatial optimization model was designed to maximize equity in accessibility to
residential care facility in Beijing, China \cite{tao_spatial_2014}. When
optimizing health accessibility, there are two competing goals: equity and
efficiency \cite{krugman_opinion_2013,meyer_equity_2008}. Equity may be defined
as equal access to healthcare for everyone \cite{culyer_equity_1993}. An
efficient situation is when everything has been done to help any person without
harming anyone else \cite{hemenway_optimal_1982}. While some argue that
efficiency should be ad-dressed in priority \cite{hemenway_optimal_1982}, others
agree that equity is a matter of ethical obligation, especially in public health
\cite{fried_rights_1975, oliver_equity_2004}. Regarding efficiency optimization,
the most popular algorithms are p-median, \ac{lscp} and \ac{mclp}. The p-median
algorithm minimizes the weighted sum of distances between users and facilities
\cite{murad_using_2021}. \ac{lscp} minimizes the number of facilities needed to
cover all demand \cite{shavandi_fuzzy_2006}. \ac{lscp} maximizes the demand
covered within a desired distance or time threshold by locating a given number
of facilities \cite{casado_heuristical_2005}. To reach equal access to
healthcare, quadratic programming has been used to  minimize the variance of
accessibility scores defined by the \ac{2sfca} \cite{wang_planning_2013}.
Similarly, a \ac{pso} algorithm was developed to minimize the total square
difference between the accessibility score of each demand location and the
weighted average accessibility score \cite{tao_spatial_2014}. Finally, a
two-step optimization algorithm has been developed to address the dual
objectives of efficiency and equality, by first choosing where to site new
hospitals and then deciding which capacity they should have
\cite{luo_two-step_2017,li_two-step_2017}. However, most of the previous
algorithms seek locations to open new health facilities. Regarding oncology
care, opening new facilities can be very costly and hard in practice.

\subsection{Patients travel}

Cancer treatment delay is a problem in health systems worldwide, increasing
mortality for many types of cancers \cite{hanna_mortality_2020}, including
breast cancer \cite{caplan_delay_1992, williams_assessment_2015,
    pace_delays_2015}. Distance between patients residence and diagnosing hospitals
is among the factors causing these delays, especially for cancer types that are
hard to diagnose \cite{flytkjaer_virgilsen_cancer_2019}. While accessibility to
healthcare is growing, research found that 8.9\% of the global population (646
million people) could not reach healthcare within one hour if they had access to
motorized transport \cite{weiss_global_2020}. Thus, a non insignificant part of
the population might be exposed to lower prognosis.

The benefits of centralized healthcare have been debated. A centralized approach
often requires patients to travel far away from their home and their local
community hospitals \cite{woo_centralisation_2012}. Patients subject to longer
travels to reach a specialized hospital are likely to be affected by the travel
burden and separation from their social environment \cite{payne_impact_2000}. In
the debate between local versus centralized healthcare provision, there are
evidence of an association between travel distance and health outcomes
\cite{kelly_are_2016}. Unsurprisingly, travel to cancer treatment is
inconvenient for some patients and might even act as a barrier to treatment
\cite{payne_impact_2000}. Research also showed that patients who lived far from
hospitals and had to travel more than 50 miles had a more advanced stage at
diagnosis, lower adherence to encoded treatments, a worse prognosis, and a worse
quality of life \cite{ambroggi_distance_2015}. More research linked travel
burden with lower treatment compliance
\cite{dutta_evaluation_2013,guidry_transportation_1997}. The distance from the
hospital influences the choice of appropriate treatment by cancer patients. In
breast cancer, patients living farther from a radiation treatment facility more
often underwent mastectomy instead of breast conservative surgery
\cite{schroen_impact_2005,celaya_travel_2006,voti_treatment_2006,meden_relationship_2002,nattinger_relationship_2001,boscoe_geographic_2011}
or did not undergo radiotherapy after breast cancer surgery
\cite{satasivam_dilemma_2014,schroen_impact_2005,celaya_travel_2006}. In non
small cell lung cancer, patients were most likely to not undergo potentially
curative surgery if they lived far from a specialist hospital and only attended
a general hospital for their care \cite{tracey_patients_2015}. Moreover, the
necessity for repeated visits for cancer diagnosis and treatment makes distance
an even more important issue for the patient\cite{guidry_transportation_1997}.
However, for hard to diagnose cancer type like rectum or testis cancers,
distance was associated with decreasing odds of advanced disease stage
\cite{virgilsen_travel_2019}. This is possibly due to being treated in more
specialized hospitals. The negative effects of centralized healthcare are even
more pronounced for patients living in rural areas. Indeed, rural cancer
patients face more challenges in receiving care, due to the limited availability
of providers and clinical trials, as well as transportation barriers and
financial issues \cite{charlton_challenges_2015}. There are evidence of poorer
treatments and outcomes for patients living in rural areas. For instance, in
Australia, poorer survival and variations in clinical management have been
reported for breast cancer women living in non metropolitan areas
\cite{dasgupta_variations_2018}. Still in Australia, breast cancer women treated
in a rural hospital had a reduced likelihood of breast conservative surgery
\cite{hall_unequal_2004}.  The hazard of death from ovarian cancer was greater
in women treated at a public general hospital than in women treated at a
gynecological oncology service \cite{tracey_effects_2014}. Contacting a
provincial hospital instead of a university hospital might lead to diagnosis and
treatment delays, which could be improved by a better referral system
\cite{thongsuksai_delay_2000}. In Australia, patients living farther from a
radiotherapy service were more likely to die of rectal cancer, with a 6\% risk
increase for each additional 100km \cite{baade_distance_2011}. In Rwanda, rural
breast cancer patients who lived in the same district as breast cancer hospitals
had a decreased likelihood of system delay \cite{pace_delays_2015}. In Canada,
place of residence seems to influence health outcomes in patients with diffuse
large B-cell lymphoma \cite{lee_effect_2014}. They found that rural and
metropolitan patients had similar survival; however, patients in small and
medium urban areas experienced worse outcomes than those in metropolitan areas.
Thus, rural culture might have a dual effect on health outcomes. On one hand,
distance, transportation, and health services shortage are barriers to
healthcare. On the other hand, rural culture comes with community belonging, and
deeper relationship with health care professionals, which might be beneficial
for some patients \cite{brundisini_chronic_2013}.

Additionally to having a negative impact on patients health, longer travels
participate in global warming due to their \ac{co2} emissions.  The World Health
Organization called climate change the greatest threat to global health in the
21st century, significantly affecting hundreds of millions of people
\cite{change_climate_2015}. The United Nations created the \ac{ipcc} to assess
the science related to climate change and provide governments with scientific
information that they can use to develop climate policies. The health care
sector is an important contributor to \ac{co2} emissions. An international
comparison of health care carbon footprints showed that, on average, the health
carbon footprint in 2014 constituted 5.5\% of the total national carbon
footprint \cite{pichler_international_2019}. Hence, the health sector has a
responsibility to take climate action
\cite{health_care_without_harm_hcwh_global_2021}. Especially since the Paris
Agreement, where countries agreed to cut \ac{ghg} emissions to keep global
warming below 2 degrees Celsius. Today, hospitals are powered by fossile energy
such as coal, oil and gas. Healthcare related travels, and the manufacture and
transport of healthcare products are also major causes of \ac{ghg} emissions.
Ultimately, all health systems will need to reach near zero emissions by 2050,
which can be more cost effective than business as usual. The Lancet Countdown on
health and climate change started to review annually the relation between health
and climate change \cite{watts_2020_2021}. A large share of these carbon
emissions is due to patients journeys
\cite{andrews_carbon_2013,nicolet_what_2022} because most patients travel by car
\cite{forner_carbon_2021}. With centralization of care, patients are encouraged
to be treated in large hospitals for better outcome \cite{eskander_health_2016}.
Such hospitals are in urban areas, and the populations living in rural areas
will have to travel longer to reach these centers, resulting in higher carbon
emissions. In France, few studies have evaluated the ecological impact of cancer
care \cite{guillon_empreinte_2020}. The Shift Project is a French think tank
that works towards a carbon-free economy. As a non-profit organization, they
inform and influence the debate on the energy transition. In 2021, the Shift
Project released a report on how to decarbonize the health care sector in France
\cite{the_shift_project_plan_2021}. They identified that most of the \ac{ghg}
emissions were scope 3 emissions, which are indirect emissions that occur in the
hospitals value chain. Among these emissions, the largest source are
pharmaceuticals and medical device buying, followed by patients and visitors
transportation. The Shift Project states that emissions related to
transportation should be cut by 99\%, through measures like increasing public
transportation and telemedicine.Telemedicine includes all medical practices that
allow patients to be treated remotely from a health facility. It has been used
increasingly around the world, even in oncology where it is sometimes referred
as teleoncology
\cite{mooi_teleoncology_2012,sabesan_are_2014,sabesan_timely_2014,sabesan_medical_2014}.
Teleoncology models have been used to provide access to specialized cancer care
for people in rural, remote and other disadvantaged areas, which minimizes the
access difficulties and disparities \cite{sabesan_telemedicine_2012,
    sabesan_are_2014}. Teleoncology models can also be beneficial in training
medical, nursing, and allied health trainees and staff at rural centers
\cite{sabesan_medical_2014}. Research reported multiple benefits of telemedicine
at every level of care, including education, prevention, diagnosis, treatment,
and monitoring \cite{bertucci_outpatient_2019}. However, besides the expected
benefits, several questions and fears are emerging
\cite{bertucci_outpatient_2019}. First, there is a risk of patient isolation,
due to the absence of in-person meeting. It is also more difficult to build an
atmosphere of trust during remote consultations and the examinations might be of
inferior quality. Finally, digital divide is a major limitation of e-health, as
certain categories of patients do not have access to the internet or to a
smartphone.

\subsection{Transparent Healthcare}

Over the past few years, there has been a massive change in the way we
communicate and interact with information. The amount of data and content
available to the public keeps increasing, as well as the number of information
delivery platforms. Studies define this phenomenon as ``the communications
revolution'' \cite{viswanath_communications_2012}. Smartphones democratization
and adoption rate are partly responsible for this revolution. Indeed, a large
and growing number of people own a smartphone, enabling them to access
information anytime and anywhere. Through this, there has been a change in how
people access and use information. With the increasing number of media sources,
mass audience is now split into smaller groups who share common characteristics
and interests. Also, the growth of online audience is now far outpacing the
other media. As a benefit of this communication revolution, it is getting easier
to access resources online, even technical resources such as technical reports
and scientific articles. While these materials may not always be intended for a
mainstream audience, their availability offers opportunities for access and
interpretation by different groups.

The healthcare sector is no exception in this revolution, and health resources
are increasingly available online
\cite{viswanath_science_2005,viswanath_communications_2012}, changing how
patients interact with health providers. Communication has been found to play a
central role in cancer prevention and control. It can provide information on
cancer prevention, monitor lifestyles and health behaviors, promote
participatory decision making during cancer detection, diagnosis, and treatment,
and foster quality of life during survivorship or end of life
\cite{viswanath_communications_2012}. When diagnosed with cancer patients and
their family members lives change radically. They receive treatments and have to
make choices with serious consequences. Such diseases and treatments are
complex, but should be understood before decisions are made. Patients and their
family members should be provided with intelligible and up to date information
on the stage of disease, treatment options and complementary therapies
\cite{butow_dynamics_1997,cassileth_information_1980}.

Multiple benefits of bringing more information to the patients have been
reported. Involving cancer patients in decision-making on their pathways
improves their satisfaction and quality of life, compliance with treatment and
their ability to manage symptoms
\cite{johnson_effects_1982,hack_feasibility_1999,mohide_randomised_1996,mcpherson_effective_2001,
    sheabudgell_information_2014,huchcroft_testing_1984,cegala_patient_2003,viswanath_science_2005}.
Moreover, medically related education interventions are most effective when they
are tailored to patients' individual needs, especially for cancer patients
\cite{cegala_patient_2003}. Through all these benefits, it is clear that
monitoring patient information seeking experiences over time is important
\cite{finney_rutten_cancer-related_2016}.

As a matter of fact, patients are often seeking information during their
pathways. In the United States of America, a survey from the Health Information
National Trends (HINTS) \cite{hesse_trust_2005} measured online health
activities, levels of trust, and source preference for 6,369 people. They
observed that physicians remained the most trusted source of information,
despite an increasing number of people looking for information online.

However, there is increasing evidence in the literature that patients are often
not satisfied with the information they received. Some reported to lack
information on their disease and its consequences
\cite{mcpherson_effective_2001}, while others forget or misunderstand the
information conveyed \cite{ley_communicating_1988,hogbin_getting_1989}. The
interaction with their physician has also been cited as a major cause of
dissatisfaction \cite{stewart_effective_1995,bartlett_effects_1984} at all
stages of illness \cite{higginson_palliative_1990}. Patients reported
insufficient time spent on communication during the clinical encounter and
physicians inability to keep up with the most current information and advances
in cancer care \cite{anderson_impact_2003}. Some patients reported incorrect
diagnosis, or not receiving the most up-to-date cancer information from their
physician, especially for rare cancers \cite{dolce_internet_2011}. Patients who
need health information but experience difficulties have been found at risk of
experiencing poorer psychosocial health \cite{arora_barriers_2002}.

A Canadian study surveyed patients attending appointments at follow-up cancer
clinics in Calgary, Alberta \cite{sheabudgell_information_2014} between 2011 and
2012. They approached 648 patients and obtained responses from 411 one of them.
The study aimed at: identifying information needs of patients when meeting their
physician for a follow-up; listing patients preferences on how to receive
information. Here are the results they gathered regarding information seeking
patterns. The most frequently reported source of information was the Internet
(57.4\%); health provider (32.6\%), brochures or pamphlets (25.1\%), and cancer
organizations (24.3\%). The most frequently reported types of information sought
included information about a specific type of cancer (43.1\%), treatment or
cures for cancer (29.4\%), prognosis or recovery from cancer (29.0\%), and
prevention of cancer (27.0\%). The least frequently reported types of cancer
information sought included where to get medical care (3.4\%), paying for
medical care or insurance (4.6\%), and cancer organizations (5.4\%). Regarding
trust, the physician or health care provider was largely the most trusted source
of information, followed by Internet, and family and friends. The least trusted
sources of information included radio, newspaper, and television.

More evidence is reported on the use of the internet for health information
retrieval
\cite{chen_impact_2001,pereira_internet_2000,ziebland_how_2004,dolce_internet_2011}.
For instance, an online questionnaire was administered to participants of
cancer-related communities hosted by the Association of Cancer Online Resources
(ACOR) \cite{dolce_internet_2011}. As a result, 488 participants shared their
personal experiences on why and how they accessed online health resources.
Participants who experienced a lack of informational support related to
procedures found blogs and testimonies online that helped them to know what to
expect from a physical and emotional perspective. Moreover, for rare diseases,
physicians might actually benefit from patients looking from additional
information online, as it could bring additional knowledge to them, and even
change their plans for care. Aware patients can also challenge their physicians
by asking meaningful questions and participate in the tailoring of their
treatment plans. Finally, online communities allowed patients to identify
physicians with a proven track record in cancer care. They endorsed care
providers who took the time to answer questions, as well as specialists from
major cancer centers, that brought superior care which led to better outcome.
Indeed, \ac{gp} play a crucial role in early cancer detection because the
majority of cancer patients initially consult their \ac{gp} with symptoms.
Therefore, the actions taken by the \ac{gp} upon the patient's symptom
presentation may considerably affect the cancer trajectory
\cite{flytkjaer_virgilsen_cancer_2019}. To sum up, the increased usage of the
Internet by cancer patients puts new demands on health care professionals.
Patients need advice about how to find reliable and credible web sites and also
help with authenticating and interpreting the information they find
\cite{carlsson_cancer_2009}.

While patients are looking for informations on their symptoms, diseases and
treatments, it would be crucial for them to know better about their physician's
ability, especially for cancer surgery. In cancer care, surgery is one of the
most important part of the treatment, and is directly linked to the surgeon
ability.

Surgeon and hospital-related factors have been found to be direct predictors of
outcome in colorectal cancer surgery \cite{renzulli_influence_2006,
    bonati_surgeon_2021}. In breast cancer, patients managed by high-volume surgeons
were more likely to have breast-conserving surgery (BCS) than those managed by
low-volume surgeons \cite{mcdermott_surgeon_2013}. Moreover, breast cancer
patients who receive treatment from experienced and specialized surgeons are
more likely to receive the standard sentinel lymph node biopsy
\cite{yen_surgeon_2014}. The surgeon's expertise and learning curve is directly
related to the patient's outcome \cite{renzulli_learning_2005}. A low surgeon or
hospital caseload may be compensated by intensified supervision or by improved
training and teaching \cite{bonati_surgeon_2021}. From all these findings, it is
questioned whether surgeons should have an ethical obligation to inform patients
of their surgical volume and outcomes \cite{glaser_surgeon_2019}. One way to
monitor the surgeons abilities is the use of quality indicators, which have been
developed in high income countries and contributed to improved quality of care
and patient outcomes over time \cite{nietz_quality_2020}.

\section{Chapters overview}

\subsubsection{Care center characterization}
In this chapter, we first proposed a method to automatically label
all the hospitals in metropolitan France, based on their statistics and
available health services. Lastly, we studied the collaborations between
the hospitals, based on patients who visited multiple hospitals during their
pathways. Through community detection algorithms, we grouped hospitals that
frequently exchange patients together. By adding the oncology specialization
label within the discovered communities, we believe we can propose new
hospital groups that are based on patient real-life data, to improve
collaborations and ultimately benefit the patients.

\subsubsection{Accessibility score}
In what follows, we applied \ac{sa} methods to
quantify the accessibility the oncology care in metropolitan France.
Intuitively, we compute a score for every municipality that measures how easy it
would be for patients living in a given municipality to reach oncology care.

\subsubsection{Accessibility optimization}
In this work, we are interested in the case where the health facilities locations are
fixed, and the only lever to improve accessibility is to increase their
capacities.Given a capacity budget, we want to know which facilities to grow and
by how much. We introduce \ac{camion}, an accessibility optimization algorithm
based on \ac{fca} and \ac{lp}. The initial accessibility score was computed with
the \ac{e2sfca} algorithm \cite{luo_enhanced_2009} but our algorithm can
generalize to more \ac{fca} derivatives. In the following sections, we proposed
two approaches for optimizing the accessibility scores. The first one is an
overall optimization, where we seek to maximize the total accessibility. The
second one is a maxi-min optimization, where we want to maximize the minimum
accessibility instead. The first approach could be seen as efficiency
maximization where the second method aims towards equity. Then, we embedded our
results and algorithms into a web application  called
``oncology-accessibility''. Through this web application, we let the users run
the optimization algorithm with the parameters they want, and visualize the
output on interactive maps and figures. We believe such an app could benefit the
healthcare professionals, to help addressing the accessibility disparities in
the country.

\subsubsection{Patients routes}
In this chapter, we analyzed the travels of cancer patients in metropolitan
France. Our goal was to assess whether the earlier observations on the negative
effects of centralization of care were happening in France. Hence, we first
described the travel duration distribution in metropolitan France, and compared
it with the population densities and the oncology specialization of the visited
hospital. Then, we argued that the negative effects of travel on cancer patients
was not only due to driving distance and duration: the road sinuosity should
also be taken into account. We proposed a travel burden index, which is a
composite indicator based on multiple variables to evaluate how easy it is to go
from a population location to an hospital. Additionally, we estimated the carbon
footprint of cancer patients travels, and compared these numbers across the
different regions. Finally, we ran an optimization algorithm to simulate the
scenario where every patient traveled to the closest hospital, such that the
hospitals capacities were not exceeded. We only considered Breast Cancer
patients as this cancer is relatively frequent, and many hospitals have the
required expertise.

\subsubsection{Transparent healthcare}
With these evidences of healthcare information needs, we developed Healthcare
Network, a web application that lists every hospital in France, and displays key
statistics on them. The application is directed to either health professionals
or patients. Health professionals might use it to gain insights about specific
hospitals, and look for the best place to send their patients when they lack
expertise. Patients could learn more about the hospital they have been sent to,
check the care quality or surgery volume.
