\chapter*{Conclusion}
\addcontentsline{toc}{chapter}{Conclusion}

\subsection*{Contributions}

We recall that the purpose of this thesis was to study the geographical
and socio-demographic disparities in oncology care pathways, in
metropolitan France.

In the first chapter, we described the hospitals
available in the country, and characterized them regarding oncology
specialization. That characterization process was automatically performed
through an unsupervised clustering algorithm, trained on hospitals statistics
from the \ac{sae} public survey. We were then able to differentiate the
most suited hospitals for oncology care, and isolate the hospitals that had
no oncology activity. Then, we studied the collaborations between these
hospitals, measured by the number of patients who visited a common hospital
during their pathways. From this collaboration dataset, we could discover
communities of hospitals that frequently exchange patients together. These
communities contain hospitals with different degree of oncology specialization.
This information could be a starting point to creating oncology collaboration
groups, consisting in hospitals working together to make sure the hospitals
with less expertise are continuously trained by more specialized hospitals.

In the next chapter, we studied the accessibility to oncology care centers in
metropolitan France. We computed an accessibility score for every municipality
in metropolitan France. The score reflects how easy it would be for patients
from a given municipality to reach an oncology specialized hospital. This score
is based on a weighting between supply and demand, as well as travel impedance.
We described the spatial distribution of this score, which was higher in dense
areas, near the most specialized hospitals, identified through the clustering
step.

Then, we proposed an optimization algorithm to identify which hospitals to grow
in order to maximize the oncology accessibility. This algorithm took as input
the current accessibility distribution, as well as some user-defined
constraints. Such constraints may include a maximum hospital growth percentage,
based on the current hospital oncology specialization. Through this optimization
process, we identified a list of hospitals that should be grown in priority to
improve the oncology accessibility distribution. The results were detailed for
every region. We packaged our method into a web application, that could be used
by healthcare professionals to run simulations and eventually improve the
healthcare planning, benefiting millions of patients.

The previous work on oncology accessibility did not directly studied the actual
cancer patients routes. In the next chapter, we extracted all the visited
hospitals during the pathways of cancer patients, and described the duration and
distance traveled based on the patients residence. These results validated our
oncology accessibility score since travel durations were longer in areas with
low accessibility scores. Longer travels were shown to have a negative impact on
the patients prognosis and treatment. Moreover, long travels often increases
patients fatigue, due to the travel burden. We argued that travel duration was
not the only factor to consider when studying the tediousness of a journey. We
built a composite indicator to reflect the travel burden of a route, based on
duration, distance and road sinuosity. We showed that patients living in rural
areas had higher travel burden, due to the longer drives they experienced, as
well as the lower road quality and higher sinuosity. Finally, we proposed an
algorithm that simulates a setup where every patient would visit the closest
specialized hospital, while making sure the hospitals capacities were not
exceeded. We showed that this approach could reduce the average driving
duration by 36\%, as well as the associated carbon footprint of the journey.

Although, in practice, patients are oriented to an hospital by their general
practitioner. There are multiple evidences in the literature that patients
are not satisfied with the level of information they receive during their
pathways. In cancer care, some patients could be sent to the wrong hospital,
without them noticing. When that is the case, the hospital could either be
a well suited hospital, but unnecessarily far from the patient residence;
or an hospital that is not experienced enough in the patients pathology. For
these reasons, we built ``healthcare-network'', a web application that lists
all the hospitals in metropolitan France, and displays key statistics on them.
The application could be used by patients to learn more about the hospitals
around them, and by health professionals, to make sure the hospital they are
sending their patients are well suited for their pathologies. We believe such
tool could incentivize physicians to send patients closer to their location
of residence. Moreover, bringing more transparency to oncology care could be
a way to reduce disparities, provided that all the population has an equal
access to these online tools.

\subsection*{Future work}

Our oncology specialization clusters could be used in further research to assess
whether the oncology care pathways are more often degraded in hospitals from the
least specialized clusters. For instance, our clusters could be the input
variables of survival analyses, to assess whether there are significant
variations in the prognosis based on the oncology specialization of the chosen
hospital. More research could also be done on the effectiveness of
collaborations between the oncology communities we discovered. These communities
are a first proposition of hospitals candidates that could work together to
better treat patients in the neighboring municipalities. Similarly, our
accessibility scores could be used in survival analyses, to assess whether
patients living in the areas with low accessibility scores have more degraded
pathways and lower prognosis. Regarding the web applications we developed,
they could be introduced to healthcare professionals in France, like the
Regional Health Agencies (ARS), responsible of the organization and the
coordination of the hospitals in the country. Working closely with these
professionals would allow to adapt our tools to their needs, so they can
eventually be used in practice to take concrete decisions on the planning of
care in the country.