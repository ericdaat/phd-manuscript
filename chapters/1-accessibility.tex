\chapter{Accessibility to oncology care}

\section{Context}

\subsection{Motivation}

While a lot of the ongoing research is focusing on finding new cancer treatments, accessibility to oncology care receives less attention. Yet, several studies have showed that access to health services plays a key role in cancer survival. For instance, geographic residency status and social environment seem to explain treatment and prognosis disparities for patients with non-small cell lung cancer \cite{johnson_treatment_2014}. In France, increases in travel times to health services were associated with lower survival rates for patients with a colorectal cancer \cite{dejardin_influence_2014}. In New Zealand, living in deprived areas, far from a cancer center or from primary care was associated with lower survival chances for patients with colorectal, lung and prostate cancers \cite{haynes_cancer_2008}.

\subsection{Accessibility definition}

Accessibility refers to the relative ease by which services can be reached from a given location \cite{wang_measurement_2012}. Accessibility can be defined by spatial factors, determined by where you are; and non-spatial factors, determined by who you are \cite{khan_integrated_1992}. Spatial accessibility methods assess the availability of supply locations from demand locations, connected by a travel impedance metric. Supply locations are characterized by their capacity or quantity of available resource. Similarly, demand locations are characterized by their population. Such methods have been successfully used to measure access to healthcare, such as primary care \cite{guagliardo_spatial_2004} or oncology care \cite{wang_measurement_2012,zahnd_spatial_2021,alahmadi_spatial_2013} in several countries including France \cite{launay_methodology_2019,gusmano_disparities_2014,gao_assessment_2016}. In what follows, we restrict accessibility to spatial accessibility and use both terms interchangeably.

\subsection{Spatial accessibility methods}

There are several ways to compute accessibility to healthcare \cite{guagliardo_spatial_2004}. The easiest and most straightforward methods are computed within bordered areas, like provider-to-population ratios in each municipality. While they are very intuitive, these methods do not account for border crossing, or travel impedance, which makes them less accurate. Recently, a new type of method has been developed and is now used in most spatial accessibility papers. This algorithm is called Two Step Floating Catchment Area (2SFCA) \cite{luo_using_2004}. It is a two-step method that first computes a provider-to-population ratio for each provider location. In the second step, for each population location, an accessibility score is obtained by summing the provider-to-population ratios. For the algorithm to work, a catchment threshold (distance or travel time) must be set. Above this threshold, a provider location is considered unreachable from the population location, and vice versa. The 2SFCA method does not account for distance decay: a care center is either reachable or not. The Enhanced Two Step Floating Catchment Area (e2SFCA) \cite{luo_enhanced_2009} addresses this limitation by applying weights to differentiate travel zones in both steps.
We now explain more formally how to compute eS2FCA scores. Consider $P_i$ the population at location $i$, with $1 \leq i \leq n$ where n is the number of population locations. Similarly, consider $S_u$ the capacity of care center $u$, with $1 \leq u \leq m$ where $m$ is the number of care centers. Finally, let $d_{iu}$ be the matrix of size $n \times m$ containing the distances between location i and care center u. We consider r sub-catchment zones each associated with a weight $W_s$, and a distance $D_s$, with $1 \leq s \leq r$, such that $D_1 D_2 < ... < D_r$ and $W_1 > W_2 > ... > W_r$. The resulting r travel intervals are $I_1=[0, D_1], I_2=[D_1, D_2 ],… ,I_r=[D_{r-1}-,D_r]$. The accessibility $A_i$ of a population location $i$ is computed in two steps:

\begin{itemize}
    \item Step 1: for every care center u, compute its weighted capacity-to-population ratio $R_u$.

    \begin{equation}
    R_u =  \frac{S_u}{\sum_{s=1}^{r} W_s \sum_{i, d_{iu} \in I_s} P_i}
    \end{equation}

    \item Step 2: for every population location, compute $A_i$ as the sum all the weighted $R_u$ of the reachable care centers.

    \begin{equation}
    A_i = \sum_{s=1}^{r} W_s \sum_{u, d_{iu} \in I_s} R_u
    \end{equation}
\end{itemize}

\section{Spatial accessibility to oncology care centers in metropolitan France}
