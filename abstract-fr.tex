\begin{mdframed}[linecolor=Prune,linewidth=1]

    \textbf{Titre:} Etude des disparités géographiques et socio-démographiques dans les parcours de soins en oncologie.

    \noindent \textbf{Mots clés:} 3 à 6 mots clefs (version en français)

    \vspace{-.5cm}
    \begin{multicols}{2}
        \noindent \textbf{Résumé:} On estime à 382 000 le nombre de nouveaux cas de
        cancers incidents et à 157 400 le nombre de décès en 2018 en France. Le Plan
        Cancer 2014-2019 annonce les objectifs à mettre en oeuvre dans la lutte contre le
        cancer en France. En particulier, les objectifs 2 et 7 insistent sur la qualité
        du parcours de soin : ils ambitionnent respectivement de ``garantir la qualité
        et la sécurité des prises en charge'' et ``assurer des prises en charge globales
        et personnalisés''. Dans le but de standardiser le parcours de soin tout en
        personnalisant la prise en charge, des trajectoires de soin ont été instaurées.
        La définition de ces trajectoires de soin optimales s'appuie sur des
        recommandations de bonnes pratiques nationales et internationales. Nous
        proposons d'étudier en détails les disparités géographiques et
        socio-démographiques dans les parcours de soin des patients atteints d'un cancer
        en France. Dans un premier temps, nous chercherons à caractériser les
        établissements de santé en France à partir de leur activité en oncologie. Puis,
        nous étudierons la distribution de ces centres sur le territoire, afin de mettre
        en évidence d'éventuelles disparités dans l'accès à ceux-ci. Enfin, nous
        tenterons de proposer un algorithme de recommandations de centres de soins, à
        partir des données du Système National des Données de Santé (SNDS). Cet
        algorithme aura pour but de guider les patients vers le centre optimal, en vue
        de maximiser la qualité du parcours de soins.

    \end{multicols}

\end{mdframed}
